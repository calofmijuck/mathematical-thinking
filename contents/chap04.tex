\chapter{증명법}

1장에서 얘기한 것 처럼 모든 수학 문제는 \textbf{명제}이고, 수학 문제를 푸는 것은 명제가 참이 되도록 가정 또는 결론을 채우는 것이다. 그런데 수학에서 어떤 명제가 참이라고 주장하기 위해서는 \textbf{논리적인 과정을 통해 그 명제가 참임을 증명}해야 한다. 사실상 \textbf{풀이 과정이 곧 증명}인 셈이다.

수학 공부를 하다 보면 개념을 공부할 때 어떤 정리나 성질에 대한 증명을 접하기도 하고, 서술형 문제를 만나 풀이 과정(증명)을 적기도 하고, 나아가 명시적으로 어떤 사실을 증명하라는 문제를 접하기도 한다.

한편, 이러한 증명들은 학교 시험에 잘 출제되지 않는 편이며, 일반적으로 서술형 문제의 비중은 객관식 문제에 비해 낮은 편이다. 나아가 이와 같은 문제들은 모의고사나 수능에 아예 출제되지도 않는다. 그러다 보니 많은 학생들이 증명의 중요성을 간과하게 된다.

수학 공부에서 증명이 중요한 이유는 간단하다. \textbf{증명을 적으며 수학적으로 성장}하기 때문이다. 증명을 적는 과정은 \textbf{배운 것을 올바르게 정리하는 과정}이다. 그러므로 증명에 사용된 개념을 더욱 명확하게 이해하게 되고, 증명이 논리적으로 올바른지 스스로 판단하는 \textbf{수학적 사고력}을 기르게 된다.

또한 증명은 \textbf{개념에 대한 깊이있는 이해}를 돕는다. 증명 속에 숨겨진 아이디어를 배우게 되고, 증명을 이해하면 해당 명제의 진정한 의미를 이해하게 되기도 한다. 이렇게 얻은 수학적 사고방식과 개념에 대한 이해는 \textbf{고난도 문제를 해결하는 강력한 도구}가 되어, 기계적인 문제 풀이와는 차원이 다른 성장을 가져다 준다.

그러나 증명을 적는 것은 결코 쉽지 않기 때문에 많은 시간과 훈련이 필요하다. 2장에서 배운 논리적 추론 규칙을 적용하여 단계적으로 사고하는 것에 익숙해지고, 모든 문제의 풀이 과정을 단계적으로 정리하다 보면 어느새 증명에 익숙해진 자신을 발견하게 될 것이다.

이번 장에서는 참인 명제를 증명하는 3가지 대표적인 방법과, 명제가 거짓임을 증명하기 위해 반례를 찾는 방법을 소개한다.

\section{직접증명법}

% \ex. (각의 이등분선의 작도)

% \begin{tikzpicture}
%     \coordinate (O) at (0, 0);
%     \coordinate (A) at (3.5, 0);
%     \coordinate (B) at (1.75, 3.031);
%     \coordinate (P) at (4.175, 2.408);

%     \draw [-,very thick] node[below left] {\(O\)} (O) -- (5, 0);
%     \draw [-,very thick] (O) -- (2.5, 4.33);

%     \draw[thick,red] ([shift=(-5:3.5)] O) arc (-5:65:3.5);

%     \draw[thick,red] ([shift=(70:2.5)] A) arc (70:80:2.5);
%     \draw[thick,red] ([shift=(-20:2.5)] B) arc (-20:-10:2.5);
%     \draw[dotted, thick] (O) -- (30:5);

%     \filldraw[fill=black] (A) circle (0.03) node[below right] {\(A\)};
%     \filldraw[fill=black] (B) circle (0.03) node[above left=2] {\(B\)};
%     \filldraw[fill=black] (P) circle (0.03) node[above right=2] {\(P\)};
% \end{tikzpicture}


\section{귀류법}

\section{수학적 귀납법}

\section{반례 찾기}
