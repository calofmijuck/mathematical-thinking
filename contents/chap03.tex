\chapter{정의와 정리}

\textbf{우리가 배우는 모든 수학 개념은 정의와 정리이다!}

정의와 정리는 우리가 수학 문제를 이해하고 해결하는데 핵심적인 역할을 한다.

\section{정의}

\textbf{정의}란 어떤 대상이나 개념의 수학적인 의미이다. 보통 새로운 개념을 배울 때, 그 개념의 의미를 명확하게 하기 위해 개념을 \textbf{정의}하고 시작한다. 예를 들면, 다음은 모두 정의이다.\footnote{정의는 곧 \textbf{약속}이기 때문에 증명이 필요하지 않다. 하지만, 왜 이렇게 정의되었는지 고민해보는 것은 개념과 정의에 대한 이해를 한 층 더 깊게 해준다.}

\begin{enumerate}
    \item 세 변의 길이가 모두 같은 삼각형을 \textbf{정삼각형}이라 한다.
    \item \(x\)의 값에 따라 참이 되기도 하고 거짓이 되기도 하는 등식을 \textbf{방정식}이라 한다.
    \item 모든 \(x\)에 대하여 \(x\)와 대응되는 \(y\)가 유일하게 존재할 때, 이 대응 관계를 \textbf{함수}라 한다.
    \item 음이 아닌 실수 \(a\)에 대하여, 제곱해서 \(a\)가 되는 실수 \(x\)를 \(a\)의 \textbf{제곱근}이라 한다.
\end{enumerate}

정의가 중요한 이유는 \textbf{정의가 곧 문제 이해의 출발점}이 되기 때문이다. 정의를 제대로 알지 못하면, 문제의 가정과 결론을 정확하게 이해하지 못해 문제의 요구사항을 이해하지 못하게 된다.

\ex. 일차방정식 \(ax + 3 = 0\)의 해가 \(x = -3\)일 때, 상수 \(a\)의 값을 구하여라.

\newpage

매우 간단한 문제이다. 방정식에 \(x = -3\)을 대입하여 \(a = 1\)임을 쉽게 찾았을 것이다. 그런데 왜 이렇게 풀었는가? \textbf{정의로 돌아가자.}

\textbf{방정식의 해}란 무엇인가? 주어진 방정식을 만족시키는 \(x\)를 \textbf{방정식의 해}라 부른다. 정의를 사용해서 다시 풀어보면 다음과 같은 과정을 거치게 된다.

\begin{itemize}
    \item (가정과 결론 파악) 주어진 문제를 명제로 나타내면 다음과 같다.
          \begin{center}
              일차방정식 \(ax + 3 = 0\)의 해가 \(x = -3\)이면 \(a = [\dots]\) 이다.
          \end{center}
    \item (가정) 일차방정식 \(ax + 3 = 0\)의 해가 \(x = -3\)이다.
    \item (정의 적용) \(a(-3) + 3 = 0\)이다.
    \item (연산) \(-3a + 3 = 0\) 이다.
    \item (삼단논법) 등식의 성질을 적용하면 \(a = 1\)이다.
\end{itemize}

대입을 한 것처럼 보이지만, 사실은 단순히 정의를 적용한 것이다. 위와 같은 문제를 `대입하여 푼다'고 공부한다면, 대입하여 푸는 방법만 배우게 되겠지만, 정의를 적용하여 푸는 방식으로 공부한다면 \textbf{정의를 적용하는 훈련}을 한 것이다. \textbf{이와 같은 사고방식은 대입하여 푸는 문제 뿐만 아니라 다른 문제를 만났을 때도 막힘 없이 문제를 해결하게 해줄 확률이 높다.}

정의가 중요한 또 다른 이유는 \textbf{정의는 곧 필요충분조건}이기 때문이다. 필요충분조건은 양 방향으로 자유롭게 사용이 가능하므로 문제 풀이에서 굉장히 유용하게 사용된다.

\bigskip

\ex. 정사각형은 직사각형임을 보여라.

\pagebreak

우선, 위 문제는 정사각형과 직사각형의 정의를 명확하게 알고 있어야 해결할 수 있다. \textbf{정의 없이는 시작조차 못하는 문제이다!} 한편, 정의를 정확하게 알고 있다면 쉽게 해결 가능하다. 정의와 논리적 추론 규칙으로부터 다음과 같은 사고를 거치게 된다.

\begin{itemize}
    \item (가정과 결론 파악) \(\underset{p}{\underline{\text{정사각형}}}\)이면 \(\underset{q}{\underline{\text{직사각형}}}\)이다. 우리의 목표는 \(p \implies q\)이다.
    \item (정의) 정사각형은 \(\underset{p_1}{\underline{\text{네 변의 길이가 모두 같}}}\)고 \(\underset{p_2}{\underline{\text{네 각의 크기가 모두 같}}}\)은 사각형이다.
    \item (정의) 직사각형은 \(\underset{p_2}{\underline{\text{네 각의 크기가 모두 같}}}\)은 사각형이다.
    \item (정의) 정의는 필요충분조건이므로, \(p \iff p_1 \wedge p_2\)와 \(q \iff p_2\)임을 안다.
    \item (필요충분조건) 가정에 \(p\)가 있으므로 \(p_1 \wedge p_2\)를 대신 사용할 수 있다.
    \item (필요충분조건) \(q \iff p_2\)이므로 \(q\) 대신 \(p_2\)를 사용한다.
    \item (그리고) \(p_1 \wedge p_2\)이면 \(p_2\)이다. 따라서 \(p \implies q\)이다.
\end{itemize}

그리고 풀이는 다음과 같이 적을 것이다.

\pf 정사각형은 정의에 의해 네 각의 크기가 모두 같은 사각형이므로, 직사각형의 정의를 만족한다. 따라서 정사각형은 직사각형이다. \qed

사고 과정의 모든 단계를 적었기 때문에 문제의 난이도에 비해 복잡해 보일 것이다. 그러나 이와 같은 사고방식은 \textbf{더 어렵고 복잡한 문제를 만났을 때 주어진 것과 구하는 것을 이해하고 문제 풀이를 시작하는데 큰 도움이 된다.}

\pagebreak

\section{정리}

참인 사실로부터 논리적 추론을 통해 얻은 또 다른 사실을 \textbf{정리}라 한다. 개념의 의미를 명확하게 하기 위해 \textbf{정의}를 했다면, 그 정의로부터 유용한 사실을 이끌어내 사용하게 되는데, 이 때 사용되는 사실이 곧 \textbf{정리}이다. 예를 들어, 다음은 모두 정리이다.\footnote{정리는 참인 사실로부터 추론을 통해 얻은 것이기 때문에 \textbf{증명이 필요}하다. 증명에 대한 자세한 내용은 다음 장에서 다룬다.}

\begin{enumerate}
    \item (등식의 성질) 모든 실수 \(a, b, c\)에 대하여 \(a = b\)이면 \(a + c = b + c\)이다.
    \item (SAS 합동조건) 두 삼각형에서 두 쌍의 변의 길이가 각각 같고 그 끼인각의 크기가 같으면 두 삼각형은 합동이다.
    \item (지수법칙) 양수 \(a\)와 자연수 \(n, m\)에 대하여 \(a^n \times a^m = a^{n + m}\)이다.
    \item (근의 공식) 이차방정식 \(ax^2 + bx + c = 0\)에 대하여 \(x = \ds \frac{-b \pm \sqrt{b^2 - 4ac}}{2a}\) 이다.
    \item (피타고라스 정리) 세 변의 길이가 \(a, b, c\)인 직각삼각형에서 \(c\)가 빗변이면, \(a^2 + b^2 = c^2\)이다.
\end{enumerate}

정리가 중요한 이유는 문제 풀이 과정에서 연결고리 역할을 하기 때문이다. \textbf{수학 문제는 명제이고, 정리 또한 명제이다.} 문제를 풀기 위해서는 문제에 주어진 명제의 가정과 결론을 이해하고, 가정으로부터 결론을 얻어내는 과정을 탐색해야 한다고 했었다. \textbf{그 탐색 과정에서 정리가 사용된다!} 삼단논법을 소개할 때, 징검다리 역할을 하는 조건을 찾기 위해 다음 두 가지를 생각하면 좋다고 했었다.

\begin{enumerate}
    \item 배운 내용 중 가정이 \(p\)인 명제가 있었나?
    \item 배운 내용 중 결론이 \(r\)인 명제가 있었나?
\end{enumerate}

정리도 명제이기 때문에, 고려 대상에 포함된다! 이로부터 세 가지 사실을 알 수 있다.

\begin{enumerate}
    \item \textbf{정리를 알고 있어야 사고를 전개해 나갈 수 있다.}
    \item \textbf{문제의 주어진 것과 구하는 것은 어떤 정리를 징검다리로 사용할지 힌트가 된다.}
    \item \textbf{정리의 가정과 결론을 정확하게 알고 있어야 어떤 징검다리를 사용할지 결정할 수 있다.}
\end{enumerate}

가정과 결론을 정확하게 알지 못하면 잘못된 결과를 얻을 수 있다. 다음 예제를 통해 살펴보자.

\newpage

\ex. 다음 `유명한' 증명에서 잘못된 부분을 찾고 그 이유를 설명하라.

\bigskip

\thm. \(a = b\)이면 \(a = 0\)이다. 즉, 두 수가 같으면 두 수는 0이다?

\pf 등식의 성질을 적용한다.
\[
    \begin{aligned}
        a = b & \implies a^2 = ab \\
              & \implies a^2 - b^2 = ab - b^2 \\
              & \implies (a + b)(a - b) = (a - b)b \\
              & \implies a + b = b \\
              & \implies a = 0
    \end{aligned}
\]
따라서 \(a = b = 0\)이다. \hfill \(\boxtimes\)

\bigskip

위의 잘못된 증명을 조금만 변형하면 \(0 = 1\)을 증명하여 모든 수가 0임을 증명할 수 있다! 이처럼 \textbf{정리의 가정과 결론을 정확하게 기억하고, 올바르게 적용하며 사고를 전개해야 참인 결론을 계속 얻어가며 문제를 올바르게 풀어나갈 수 있다.}
