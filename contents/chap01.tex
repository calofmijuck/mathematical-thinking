\chapter{수학 문제란 무엇인가?}

수학 문제들을 자세히 살펴보면, 공통점을 하나 발견할 수 있다. 이는 모든 수학 문제가 곧 \textbf{명제}라는 점이다.\footnote{참이나 거짓을 분명하게 판별할 수 있는 문장이나 식을 \textbf{명제}라고 한다.} 참/거짓을 판별하는 문제들은 당연히 명제의 형태이다. 그렇지 않은 형태의 문제들도 자세히 살펴보면 모두 명제이다. 다음 예제들을 하나씩 살펴보자.

\bigskip

\begin{enumerate}
    \item \(x + y = 2\), \(xy = 1\)일 때, \((x - y)^2\)의 값을 구하여라.
    \item 이차방정식 \(x^2 + ax + 1 = 0\)이 실근을 갖도록 하는 실수 \(a\)의 범위를 구하여라.
    \item \(\sqrt{110+x}\)가 자연수가 되도록 하는 \(x\)의 값 중 가장 작은 자연수를 구하여라.
    \item 직각삼각형에서 빗변이 아닌 두 변의 길이가 각각 \(3, 4\)일 때 빗변의 길이를 구하여라.
\end{enumerate}

\bigskip

언뜻 보기에는 명제가 아닌 것처럼 보일 수 있다. 이는 대부분의 문제들이 \textbf{명제의 가정 또는 결론 부분을 비워두고 있기 때문}이다. 다시 한 번 살펴보자.

\bigskip

\begin{enumerate}
    \item \(x + y = 2\), \(xy = 1\)일 때, \((x - y)^2\)의 값은 [...]이다.
    \item 실수 \(a\)에 대하여 [...]이면 이차방정식 \(x^2 + ax + 1 = 0\)이 실근을 갖는다.
    \item \(\sqrt{110+x}\)가 자연수가 되도록 하는 \(x\)의 값 중 가장 작은 자연수는 [...]이다.
    \item 직각삼각형에서 빗변이 아닌 두 변의 길이가 각각 \(3, 4\)일 때 빗변의 길이는 [...]이다.
\end{enumerate}

\bigskip

\textbf{가정과 결론이 보인다!} 다시 보니 모두 명제임이 분명하다. 사실 문제들은 우리에게 \textbf{주어진 명제가 참이 되도록 가정 또는 결론을 채우길 요구}하고 있었던 것이다!

\begin{center}
    [수학 문제를 푼다] = [주어진 명제가 참이 되도록 가정 또는 결론을 채운다]
\end{center}

\bigskip

\pagebreak

그렇다면 수학 문제를 올바르게 풀기 위해서는 \textbf{`명제'에 대해 잘 이해하고 있어야 할 것이다.}

\begin{enumerate}
    \item \textbf{가정}으로 무엇이 주어졌는지 파악하고,
    \item \textbf{결론}으로 무엇을 요구하는지 파악해야 하며,
    \item \textbf{가정으로부터 결론을 얻어내는 과정을 탐색}해야 한다.
\end{enumerate}

문제를 읽고 이해했다면, 가정과 결론을 올바르게 파악하는 방법은 적당한 훈련을 통해 충분히 익힐 수 있다. 그러나 여기서 \textbf{(3)번의 과정이 매우 어렵다.} 바로 학생들이 `어떻게 푸는지 모르겠어요'라고 말하는 경우이다. 가정이나 결론을 파악했더라도, \textbf{문제의 요구사항을 구하기 위한 구체적인 과정}을 알아내는 것은 결코 단순하지 않기 때문이다.

\medskip

고민 끝에 아쉽게도 문제를 풀지 못하고 해설을 보게 되면, 문제의 가정으로부터 논리적인 단계를 밟아 결론과 답을 얻어내는 과정을 확인할 수 있다. 경우에 따라서는 번뜩이는 생각이나 사고의 전환이 필요하기도 하지만, 대부분의 경우 배운 내용을 적용하여 문제를 해결하고, 특별한 발상을 필요로 하지 않는다. 해설을 읽고 나면 `나는 분명 다 배웠고 아는 내용인데 왜 이런 생각을 하지 못했나?'하는 생각이 저절로 든다.

\medskip

그렇다면 \textbf{문제 해결 능력}은 어떻게 기를 수 있을까? 이를 위해 다음을 연습해야 한다.
\begin{itemize}
    \item \textbf{문제의 가정(주어진 것)과 결론(구하는 것)을 정확히 파악하는 연습}
    \item \textbf{문제의 가정과 결론을  배운 내용을 제대로 활용하는 연습}
    \item \textbf{문제의 가정과 결론을 바탕으로 해결 전략을 세우는 연습}
\end{itemize}
위 내용을 연습하기 위해 특별히 이 책의 예제들을 모두 쉬운 난이도의 문제들로 구성했다. 그러므로 문제의 답을 구하는 것에 주목하기 보다는 \textbf{생각하고 해결하는 과정에 집중}하여 올바른 수학 공부 방법과 문제 해결 전략에 대한 일반적인 방법론을 배우는 것에 집중해야 한다.

\medskip

처음에는 시간이 조금 걸리겠지만, 익숙해지면 \textbf{복잡하고 어려운 문제를 만나더라도 길을 잃지 않고 결론에 도달하는 강력한 사고력}을 가진 자신을 발견하게 될 것이다.

\pagebreak
