\chapter{논리적 추론 규칙}

이번 장에서는 자주 사용되는 논리적 추론 규칙을 다룬다.

명제 \(p \ra q\)가 참일 때, \(p \implies q\) 또는 다음과 같이 표기한다.
\[
    \begin{array}{c|l}
        (1) & p \\\hline & q
    \end{array}
\]
가로선 \(-\)를 기준으로 위에 적힌 것이 가정이고, 아래에 적힌 것이 결론이다. (1)은 가정에 번호를 붙인 것이다. 예를 들어, 가정이 여러 개여서 명제 \(p_1, p_2, p_3 \ra q\)가 참이라면, 다음과 같이 표기한다.
\[
    \begin{array}{c|l}
        (1) & p_1 \\ (2) & p_2 \\ (3) & p_3 \\\hline & q
    \end{array}
\]

\pagebreak

\section{명제 사용하기}

가장 기본적인 추론 규칙이다.
\[
    \begin{array}{c|l}
        (1) & p \ra q \\ (2) & p \\\hline & q
    \end{array}
\]
\(p \ra q\)가 참임을 아는 상황에서, \(p\)가 참이라면, 당연히 \(q\)가 참일 것이다.

문제에서는 가정 \(p\)와 결론 \(q\)만 주어지는 경우가 많기 때문에, 학생들은 \(p \ra q\) 형태의 명제를 배운 적이 있는지 찾아야 할 것이다. 해당 형태의 명제를 찾았다면, 그 명제를 사용하여 바로 결론을 얻고 문제를 해결할 수 있다.

\bigskip

다음 예제에 대하여 가정 \(p\)와 결론 \(q\)를 찾고, 사용한 명제 \(p \ra q\)를 서술하여라.

\ex. \(x = 2\)일 때, \(x + 3\)의 값을 구하여라.

\vspace*{250px}

\ex. 한변의 길이가 3인 정삼각형의 넓이를 구하여라.

\pagebreak

\section{삼단논법}

수학 전반에서 가장 많이 사용되는 추론 규칙이다.
\[
    \begin{array}{c|l}
        (1) & p \ra q \\ (2) & q \ra r \\ (3) & p \\\hline & r
    \end{array}
\]

당연한 것 같지만, 왜 그런지 살펴보자.

\pf \(p \ra q\), \(q \ra r\)이 모두 참임을 아는 상황에서, \(p\)가 참이라고 하자. 그러면 [명제 사용하기]를 가정 (1), (3)에 적용하면 \(q\)가 참임을 알 수 있다. 그러면 현재 상태는 다음과 같다.
\[
    \begin{array}{c|l}
        (1) & p \ra q \\ (2) & q \ra r \\ (3) & p \\ (4) & q \\\hline & r
    \end{array}
\]
[명제 사용하기]를 가정 (2), (4)에 적용하면 \(r\)이 참임을 알 수 있다. \qed

마찬가지로, 문제에서는 가정 \(p\)와 결론인 \(r\)만 주어지는 경우가 많다. \textbf{그러므로 결론인 \(r\)을 얻기 위해 학생들은 \(p\)와 \(r\)의 징검다리 역할을 하는 \(q\)를 찾아야 한다.} 단, \(p \ra q\), \(q \ra r\)이 모두 참인 \(q\)를 찾아야 한다.

이와 같은 \(q\)를 찾기 위해서는 다음 2가지를 생각하면 좋다.
\begin{enumerate}
    \item 배운 내용 중 가정이 \(p\)인 명제가 있었나?
    \item 배운 내용 중 결론이 \(r\)인 명제가 있었나?
\end{enumerate}

어려운 문제에서는 가정과 결론의 징검다리 역할을 하는 \(q\)를 여러 개 찾아야 하는 경우도 있다. 이 경우에도 결국에는 문제에 주어진 가정과 결론을 고려하여 그와 관련된 \(q\)를 선택하면 된다.

\pagebreak

다음 예제에서 가정 \(p\)와 결론 \(r\)을 찾고, 징검다리 \(q\)와 사용한 명제 \(p \ra q\), \(q \ra r\)을 서술하여라. 필요하다면 징검다리를 여러 번 사용해도 좋다!

\ex. \(a = 1 + \sqrt{3}\) 일 때, \(a^2 - 2a + 1\)의 값을 구하여라.

\vspace*{350px}

\ex. 방정식 \(3x + 2 = 8\)을 풀어라.

\pagebreak

\section{대우 사용하기}

명제와 그 대우는 참/거짓을 같이 한다. 명제가 참이면 대우도 참이고, 명제가 거짓이면 대우도 거짓이다. 이를 다음과 같이 표현할 수 있다.
\[
    (p \ra q) \implies (\sim q \ra \sim p)
\]
삼단논법을 사용하여 다음이 성립함을 보여라.
\[
    \begin{array}{c|l}
        (1) & p \ra q \\ (2) & \sim q \\\hline & \sim p
    \end{array}
\]

\pagebreak

\section{그리고}

\textbf{\(p\)와 \(q\)가 모두 참인 경우에만} `\(p\) 그리고 \(q\)'가 참이다. 기호로는 \(p \wedge q\)로 나타낸다.
\[
    \begin{array}{c|l}
        (1) & p  \\ (2) & q \\\hline & p \wedge q
    \end{array}
\]
`\(p\) 그리고 \(q\)'를 증명해야 하는 경우, \textbf{\(p\)와 \(q\)가 모두 성립함을 보이면 된다.}

또한 `그리고'의 정의로부터 다음이 성립한다.
\[
    \begin{array}{c|l}
        (1) & p \wedge q \\\hline & p
    \end{array}
    \qquad
    \begin{array}{c|l}
        (1) & p \wedge q \\\hline & q
    \end{array}
\]
문제에서 조건이 여러 개 주어진 경우 그 중 하나만 택하여 사용할 수도 있다는 의미이다.

\bigskip

다음 예제에서 \(p \wedge q\) 형태의 조건을 사용해 명제로 나타내고, 풀이에 사용한 명제를 서술하여라.

\ex. 2의 배수이고 3의 배수이면 6의 배수임을 보여라.

\vspace*{180px}

\ex. 연립방정식 \(\begin{cases}
    x + y = 3 \\ 2x - y = 0
\end{cases}\)을 풀어라.

\pagebreak

\section{또는}

\textbf{\(p\)와 \(q\) 중 적어도 하나가 참인 경우에만} `\(p\) 또는 \(q\)'가 참이다. 기호로는 \(p \vee q\)로 나타낸다. 정의로부터 \(p\)와 \(q\) 중 하나가 참이면 `\(p\) 또는 \(q\)'는 자동으로 참이므로 다음이 성립한다.
\[
    \begin{array}{c|l}
        (1) & p \\\hline & p \vee q
    \end{array}
    \qquad
    \begin{array}{c|l}
        (1) & q \\\hline & p \vee q
    \end{array}
\]

\subsection{경우 나누기 규칙}

다음 규칙은 주로 \textbf{경우를 나눌 때} 사용한다. `\(p_1\) 또는 \(p_2\)'가 참임을 안다고 하자. \(p_1\)이 참일 때 \(q\)가 참이고, \(p_2\)가 참일 때도 \(q\)가 참이라면, \(q\)는 참일 것이다.
\[
    \begin{array}{c|l}
        (1) & p_1 \vee p_2 \\ (2) & p_1 \ra q \\ (3) & p_2 \ra q \\\hline & q
    \end{array}
\]
\textbf{\(p_1\), \(p_2\)가 각각 참인 경우로 나누었을 때, 모든 경우에 대해 \(q\)가 성립하기 때문이다.}

단, \textbf{경우를 빠짐없이 나누었는지 확인해야 한다.} 예를 들어, \(x\)가 실수인 경우 `\(x > 0\) 또는 \(x < 0\)'으로 경우를 나누게 되면 \(x = 0\)인 경우를 빠뜨리게 되므로 주의해야 한다. 그리고 경우를 2개 이상으로 나눈다고 하더라도 각 경우마다 \(q\)가 참임을 보이면 될 것이다.

문제는 주로 \(p \ra q\) 형태로 주어지기 때문에 \(p = p_1 \vee p_2\)이면서 \(p_1 \ra q\), \(p_2 \ra q\)가 참인 \(p_1\), \(p_2\)를 찾아야 할 것이다. 다음 예제에서 조건 \(p\), \(p_1\), \(p_2\), \(q\)를 찾아보라.

\bigskip

\ex. 자연수 \(n\)에 대하여 \(n(n + 1)\)은 짝수이다.

\pagebreak

위 규칙을 강화한 다음 규칙도 종종 사용된다. 앞 규칙과 비교하면, (2)번과 (3)번 조건의 결론 부분이 다르고 결론도 달라진 것을 확인할 수 있다.
\[
    \begin{array}{c|l}
        (1) & p_1 \vee p_2 \\ (2) & p_1 \ra q_1 \\ (3) & p_2 \ra q_2 \\\hline & q_1 \vee q_2
    \end{array}
\]
`\(p_1\) 또는 \(p_2\)'가 참이라고 하자. 여기서 \(p_1 \ra q_1\), \(p_2 \ra q_2\)가 참이라면, \textbf{\(p_1\), \(p_2\) 중 적어도 하나는 참이므로 \(q_1\)과 \(q_2\) 중 적어도 하나가 참}이다. 따라서 \(q_1 \vee q_2\)가 참이어야 한다.

위와 같은 형태의 경우 나누기는 주로 \textbf{각 경우에 대해 다른 전략을 사용하고 그 결과를 합쳐야 하는 경우} 사용된다. 다음 예제에서 조건 \(p\), \(p_1\), \(p_2\), \(q_1\), \(q_2\)를 찾아보라.

\bigskip

\ex. 방정식 \((k + 2)x^2 + 2(k + 3)x + (k + 6) = 0\)이 실근을 가질 때, 실수 \(k\)의 범위를 구하여라.

\pagebreak

\subsection{경우 제거하기 규칙}

`또는'과 관련된 마지막 규칙은 \textbf{경우 제거하기 규칙}이다. `\(p\) 또는 \(q\)'가 참일 때, \(p\)가 아니라면 \(q\)가 참이어야 할 것이다. 물론 그 반대의 경우도 당연히 성립한다.
\[
    \begin{array}{c|l}
        (1) & p \vee q \\ (2) & \sim p \\\hline & q
    \end{array}
    \qquad
    \begin{array}{c|l}
        (1) & p \vee q \\ (2) & \sim q \\\hline & p
    \end{array}
\]
예를 들어, 풀이 과정에서 `\(p\) 또는 \(q\)'가 참임을 알았을 때, \(p\)가 논리적으로 모순이거나 문제의 조건에서 \(p\)를 제외하라고 했다면 \(q\)만 고려하면 된다는 의미이다. 다음 예제에 적용해 보라.

\bigskip

\ex. 좌표평면 위의 두 점 \(\rm{A}(1, -1)\), \(\rm{B}(1, 1)\)에 대하여 일차함수 \(y = ax\)가 선분 \(\overline{\rm{AB}}\)와 만나도록 하는 양수 \(a\)의 범위를 구하여라.

\pagebreak

\section{필요충분조건}

\textbf{\(p \implies q\) 그리고 \(q \implies p\)인 경우}, 간단히 \(p \iff q\)로 표기하며 \(p\)는 \(q\)이기 위한 필요충분조건이라 한다. 예를 들어, 다음과 같은 명제가 필요충분조건 형태의 명제이다.
\begin{center}
    실수 \(a, b\)에 대하여, \(ab = 0 \iff a = 0\) 또는 \(b = 0\).
\end{center}

\(p \iff q\)의 정의로부터
\begin{center}
    \(p \implies q\) 그리고 \(q \implies p\)
\end{center}
이다. 즉, 필요충분조건인 명제는 양방향으로 사용이 가능하다!

실제로 문제를 푸는 과정에서는 \(p\)가 가정으로 주어졌을 때, \(p \iff q\)인 명제를 알고 있다면 \(q\)도 가정에 추가하여 사용할 수 있게 된다.

\bigskip

\ex. 이차방정식 \(x^2 + 3x + 2 = 0\)을 풀어라.

\pagebreak

\section{모든}

\textbf{주어진 성질 \(p(x)\)가 모든 \(x\)에 대해 성립할 때}, 다음과 같이 표현한다.
\begin{center}
    \textbf{모든} \(x\)에 대하여 \(p(x)\)이다.
\end{center}
앞 부분과 달리 \(p\)에 변수 \(x\)가 들어간 것은 조건이 \(x\)를 변수로 포함하고 있기 때문이다. 예를 들어,
\begin{center}
    모든 자연수 \(n\)에 대하여 \(n > 0\)이다.
\end{center}
와 같은 명제를 생각할 수 있을 것이다.

`모든'을 포함한 명제는 말 그대로 모든 경우에 성립하므로 굉장히 강력한 명제이다. 강력한 명제는 다양한 곳에서 사용되기 때문에 반드시 기억해야 한다. 단, 여기서 \textbf{`모든'의 범위가 어디인지 주의}해야 한다. 명제 `모든 \textbf{정수} \(n\)에 대하여 \(n > 0\)이다'는 거짓이다!

`모든'을 포함한 명제를 활용하는 방법은 간단하다. \textbf{명제가 제한한 범위에 해당하는 \(x\)를 가져오면 \(p(x)\)가 참임을 사용할 수 있게 된다!} 다음과 같이 변수가 2개인 경우도 살펴보자.
\begin{center}
    (모든) 음이 아닌 실수 \(x, y\)에 대하여, \(\sqrt{xy} = \sqrt{x}\sqrt{y}\)이다.
\end{center}
\(\sqrt{3}\sqrt{5} = \sqrt{15}\)와 같은 계산을 할 때 `모든' 형태의 명제를 사용하고 있었던 것이다. 위에서 \(x = 3\), \(y = 5\)로 둔 경우이다. 변수가 3개인 경우도 마찬가지이다. 다음 예제를 통해 직접 확인해 보라.

사용한 `모든' 형태의 명제를 서술하고, 명제의 각 변수를 어떤 값으로 두었는지 명시하여라.

\bigskip

\ex. \(3^2 \times 3^4 = 3^6\)인 이유를 설명하여라.

\pagebreak

\section{어떤}

\textbf{주어진 성질 \(p(x)\)를 만족하는 \(x\)가 존재할 때}, 다음과 같이 표현한다.
\begin{center}
    \textbf{어떤} \(x\)에 대하여 \(p(x)\)이다.
\end{center}
예를 들어, 다음과 같은 명제가 `어떤'을 포함한 명제이다.
\begin{center}
    어떤 정수 \(x\)에 대하여 \(x^2 - 4x - 5 = 0\)이다.
\end{center}
`모든'과 `어떤'을 모두 사용한 형태도 가능하다.
\begin{center}
    모든 짝수 \(n\)에 대하여, 어떤 자연수 \(k\)에 대하여 \(n = 2k\)이다.
\end{center}

`어떤'을 포함한 명제는 조건 \(p(x)\)를 만족하는 \(x\)가 적어도 하나 존재함을 말해준다. 즉, \textbf{존재성}과 관련된 조건이 곧 `어떤' 형태의 명제이다. `모든'을 포함한 명제의 경우와 마찬가지로, \textbf{조건을 만족하는 \(x\)가 어떤 범위의 \(x\)인지 주의}해야 한다. 명제 `어떤 \textbf{실수} \(x\)에 대하여 \(x^2 < 0\)이다'는 거짓이다!

`어떤'을 포함한 명제를 활용하는 방법 또한 어렵지 않다. \textbf{조건을 만족하는 대상을 적당한 문자 \(x\)로 두고 \(p(x)\)가 참임을 이용한다.}

다음 예제에서 사용한 `어떤' 형태의 명제를 서술하고, 명제의 조건을 만족하는 대상을 어떤 문자로 두었는지 명시하여라.

\bigskip

\ex. 서로 다른 두 양의 실근을 갖는 이차방정식 \(x^2 + px + 2 = 0\)에 대하여 두 근의 비가 \(1 : 2\)일 때, 상수 \(p\)의 값을 구하여라.

\pagebreak
